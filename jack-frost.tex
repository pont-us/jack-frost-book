\documentclass[12pt,a4paper]{article}

\usepackage{datetime}
\newdateformat{sanedate}{%
\THEDAY{} \monthname[\THEMONTH] \THEYEAR}
\sanedate

\usepackage{ifpdf}
\usepackage[left=2cm,top=2cm,right=2cm,bottom=2cm,nohead]{geometry}
\usepackage{url}

\usepackage[T1]{fontenc}
\usepackage{lmodern}
\renewcommand{\rmdefault}{ugm} % garamond
% \usepackage[garamond]{mathdesign}

\frenchspacing

\title{The Jack Frost Tune-Book}
\author{\small Set down by Pontus Lurcock}
% \date{\normalsize 1 November 2008}

\ifpdf
% The licence file will be copied from includes/ to out/
% by the makefile.
\usepackage{xmpincl}
\includexmp{licence}
\pdfinfo
{ /Title (The Jack Frost Tune-Book)
  /Author (Pontus Lurcock)
  /CreationDate (D:20081020000000)
  /ModDate (D:\pdfdate)
  /Keywords (morris dancing)
}
\fi

\makeatletter
% no bold, thank you
\renewcommand\section{\@startsection {section}{1}{\z@}%
                                   {-3.5ex \@plus -1ex \@minus -.2ex}%
                                   {2.3ex \@plus.2ex}%
                                   {\normalfont\Large}}
\renewcommand\subsection{\@startsection {subsection}{1}{\z@}%
                                   {-3.5ex \@plus -1ex \@minus -.2ex}%
                                   {2.3ex \@plus.2ex}%
                                   {\normalfont\large\itshape}}
\makeatother

\begin{document}
\maketitle
\samepage

\section*{Introduction}

This book is chiefly intended as an aid to new and visiting musicians
at Jack Frost Morris. It's possible that it will evolve into something
more extensive, perhaps containing more details of the dances, but the
main intent is that it provide a faster alternative to learning the
Jack Frost repertoire by ear.

I have written all the tunes down from memory, and there will
doubtless be some errors and omissions; in any case it's impossible to
provide a perfect snapshot of a living tradition. But I hope that this
collection will nevertheless prove useful.

The textual parts of this work were typeset in \LaTeX{}, and the tunes
in GNU Lilypond.

\subsection*{Notation}

I have used standard musical notation as far as possible. The main
addition is the method for indicating repeats of the A and B parts: I
have simply set the number of repeats next to the mark for the part in
the music, so that, for example, `A 1' and `B 3' in {\it Dilwyn} mean
`play the A part once, and the B part thrice'. Varying repetitions are
noted as `A *' or `B *' and explained in the accompanying text.

When there is a song, it is sung {\it a capella} by the whole side at
the beginning, leading straight into the A part. Where there is no
song, the A part is usually played through once as an introduction.

Some tunes clearly go too low for a descant recorder or D whistle; in
this case there is nothing for it but to go up an octave, or
substitute something reasonable for the impossible notes.

At present the overall structure of the tunes (i.e. the total number
of repeats, variations in repetitions, and introductions for dances
without songs) is not always given, and chords are not noted, expect
for Dilwyn; I hope to remedy these shortcomings in a future edition.

\subsection*{Copyright}

The tunes, of course, are traditional, and belong to everyone. The 
transcriptions are my own, however, and are governed by the following terms.

This work is licensed under the Creative Commons
Attribution-Noncommercial-Share Alike 3.0 New Zealand License. To view
a copy of this license, visit
\url{http://creativecommons.org/licenses/by-nc-sa/3.0/nz/} or send a
letter to Creative Commons, 171 Second Street, Suite 300, San
Francisco, California, 94105, USA.

In summary, the licence states that you may copy, distribute,
transmit, and adapt this work freely, provided that you attribute the
work and do not use it for commercial purposes. And if you alter,
transform, or build upon this work, you may distribute the resulting
work only under the same or similar license to this one.

\pagebreak[2]

\section{Dilwyn (Border)}
\lilypondfile[noindent]{tunes/dilwyn.ly}

\noindent Notes: Jack Frost's signature dance. The lyrics may (reluctantly) be
bowdlerized to `My friend Billy's got a ten-foot maypole' for genteel
audiences. The volta marked `final' is only used right at the end of
the dance (when the dancers jump out at the end of the star
figure).

\pagebreak[2]

\section{Bean-setting (Badby)}
\lilypondfile[noindent]{tunes/bean-setting.ly}
\nopagebreak
\noindent Notes: the song is original to Jack Frost. Two whole-bar
rests (another Jack Frost innovation) are marked as an ossia for bars
17 and 18; these are substituted during the three repititions of the
final bout of sticking (`windscreen wipers').

\nopagebreak
%\pagebreak[1]
\section{Hindley Street (Border?)}

\lilypondfile[noindent]{tunes/hindley-street.ly}

\begin{minipage}[t]{6cm}
\vspace{0pt}
\begin{tabular}{rrl}
Verse & Repeats & Name \\
1 & 1 & Across \\
2 & 2 & Diagonals \\
3 & 1 & Star \\
4 & 2 & In and out \\
5 & 1 & Semi-left \\
6 & 2 & Spirals \\
7 & 2* & Final figure
\end{tabular}
\end{minipage} \hspace{3mm} \begin{minipage}[t]{10cm}
\vspace{0pt}
\noindent Notes: also known as the Air Traffic Controller Dance.
The A-part is always played once; the number of repeats for the
B-part varies from verse to verse, as shown in the table.

\smallskip

*In the final figure, the B-part is repeated twice when the usual four
dancers are dancing.  If two sets of four are dancing, the final
figure combines the sets. This takes longer and will need three
repeats.
\end{minipage}
\pagebreak[1]

\section{The Black Joke (Bledington)}

\lilypondfile[noindent]{tunes/the-black-joke.ly}

\pagebreak[1]

\section{Young Collins (Bledington)}

\lilypondfile[noindent]{tunes/young-collins.ly}

\noindent Notes: the song is original to Jack Frost.

\pagebreak[2]

\section{Lads a-Buncham (Adderbury)}

\lilypondfile[noindent]{tunes/lads-a-buncham.ly}

\noindent Notes: the dance finishes with a {\it double time} section,
consisting of a single repeat of the B part played at double
speed, or as near thereto as is practicable for the musicians
and dancers.

\pagebreak[2]

\section{Horsham Butchers (Adderbury*)}

\lilypondfile[noindent]{tunes/horsham-butchers.ly}

\noindent Notes: the dance finishes with a {\it double time} section,
consisting of a single repeat of the B part played at double
speed, or as near thereto as is practicable for the musicians
and dancers. The song is original to Jack Frost.

* This is `not a true Adderbury dance but danced fully to the
Adderbury tradition' (S.~Vallance of Vintage).

\pagebreak[2]

\section{Upton-upon-Severn stick dance}

\lilypondfile[noindent]{tunes/upton-upon-severn.ly}

\noindent Notes: there is also an Upton-upon-Severn handkerchief dance, but 
this of course has no place in Jack Frost's repertoire.

\pagebreak[2]

\section{Constant Billy (Headington)}

\lilypondfile[noindent]{tunes/constant-billy.ly}

\pagebreak[4]

\section{Woodhouse Bog (Border)}

\lilypondfile[noindent]{tunes/woodhouse-bog.ly}

\noindent Notes: no `special' repetition is required as an
introduction: the musicians simply play the A part twice and B part
twice as usual. The first couple dance on during the first playing of
the B part, and commence sticking at its repeat. At the end of
the dance, the last couple stick during the first playing of the B
part, and dance off during the repeat.

There is an unofficial song, but it is probably best left
as an oral tradition.

\section{The Muffin Man (Border)}

\lilypondfile[noindent]{tunes/the-muffin-man.ly}

\begin{minipage}[t]{7cm}
\vspace{0pt}
\begin{tabular}{rrl}
Verse & Repeats & Name \\
1 & 1 & Across \\
2 & 1 & Diagonals \\
3 & 2 & Swoops \\
4 & 2 & Hay with musician \\
5 & 1 & Rounds \\
\end{tabular}
\end{minipage} \hspace{3mm} \begin{minipage}[t]{9cm}
\vspace{0pt}
\noindent Notes: The B-part is always played twice; The number of
repeat for the A part varies from verse to verse, as shown in the
table.
\end{minipage}

\medskip

\noindent {\it The Muffin Man} is often danced as the last dance of a
set, as it segues quite nicely into an elegant departure from the
scene for the whole side.  The lead musician plays at the centre of
the set and takes part in the dance; any other musicians play at the
head of the set as usual. At the end of the final figure ({\it
  rounds}), the lead musician leads the dancers and other musicians
wherever he or she pleases; the music continues until the lead
musician stops playing, indicating that the dance is finished and the
side has reached its destination (generally the nearest bar).

\newenvironment{dance}[3]
{\begin{minipage}[t][60mm]{53mm}
\smallskip\small
{\bf #1}\\
#2 dancers / #3
\vspace{-2mm}
\begin{tabbing}
}
{\end{tabbing}\end{minipage}}

\noindent \begin{tabular}{|l|l|l|}

\hline
\begin{dance}{Upton-upon-Severn}{6}{1 long stick}
Back to back right \= \kill
Dance around \> {\it Doubles} \\
Back to back right \> {\it Singles} \\
Back to back left \> {\it Staves} \\
Hay right \> {\it Doubles} \\
Hay left \> {\it Singles} \\
Middles-out hay \> {\it Staves} \\
Rounds, whalebacks to finish \\
\end{dance}
&
\begin{dance}{Bean-setting (Badby)}{6}{1 long stick}
{\it Start in tight formation}\\
Back-to-back \= \kill
Double cast \> {\it Butts-tips} \\
Set straight \> {\it Figure-eights} \\
Back-to-back \> {\it Errol Flynns} \\
Hands around (sticks out) \\
\> {\it Windscreen wipers} \\
Whole hay\\
{\it Finish in shite formation}\\
\end{dance}
&
\begin{dance}{Lads a-Buncham (Adderbury)}{6}{1 long stick}
Adderbury hay. \= \kill
{\it Song and walk-round} \\
Foot up (twice) \>{\it Doubles} \\
Back to back \>{\it Singles} \\
Process down \>{\it Overheads} \\
Process up \>{\it Doubles} \\
Hands around \>{\it Singles} \\
Adderbury hay \>{\it Doubles} \\
{\it Double time to finish}
\end{dance}
\\ \hline
\begin{dance}{Muffin Man (Border)}{4}{1 long stick}
Across \\
Diagonals \\
Swoops \\
Hay with musician \\
Rounds and dance off
\end{dance}
&
\begin{dance}{Dilwyn (Border)}{4}{1 long stick}
{\it Start with song and sticking} \\
Squire's route \\
Bagman's route \\
Haybacks \\
Rats, Star \\
Jump out to finish
\end{dance}
&
\begin{dance}{Horsham Butchers (Adderb${}^{\mbox{\footnotesize\itshape y}}$)}{6}{1 long {\itshape\&} 1 short stick}
Song \\
Foot up (twice)\\
Back to back \\
Process down \\
Process up \\
Hands around \\
Adderbury hay \\
Double time to finish \\
\end{dance}

 \\ \hline 

\begin{dance}{Young Collins (Bledington)}{6}{1 long stick}
Whole gyp, \= \kill
Foot up and down \hspace{1ex} {\it Plain}\\
Half gyp \> {\it Syncopated}\\
Whole gyp \> {\it Syncopated}\\
Rounds \> {\it Tossing}
\end{dance}
&
\begin{dance}{Black Joke (Bledington)}{6}{1 long stick}
Foot up and down \\
Half gyp \\
Whole gyp \\
Rounds \\
\\
Sticking: P R L P, P L R P
\end{dance}
&
\begin{dance}{Constant Billy (Headington)}{6}{2 short sticks}
Song \\
Foot up \\
Back to back \\
Crossover \\
Whole hay to finish \\
\end{dance}
\\ \hline
\begin{dance}{Woodhouse Bog (Border)}{8}{1 short stick}
{\it First couple on, backhand sticking}\\
Swing for 2 \\
Star for 4 \\
Hay for 6 \\
Motorcycle Hay for 8 \\
Hay for 6 \\
Star for 4 \\
Swing for 2\\
{\it Last couple forehand sticking \& off}
\end{dance}
&
\begin{dance}{Hindley Street (Border)}{4+}{2 short sticks}
{\it Foot up (chorus)} \\
Across \\
Diagonals \\
Star \\
In and out \\
Semi-left \\
Spirals (bagman leads) \\
Final figure
\end{dance}
&
\begin{dance}{Hindley Street (NWA)}{4}{2 short sticks}
Foot up \\
Crossover (Across) \\
Star \\
Change (Diagonals) \\
Forward and back (In and out) \\
Spirals (Squire leads) \\
Left and Right (Semi-left) \\
Final figure
\end{dance}
\\ \hline
\end{tabular}

\end{document}
